\documentclass[a4paper,12pt]{article} % тип документа

% report, book

\usepackage{indentfirst} % Красная строка

\usepackage{listings} % code import

% Рисунки
\usepackage{graphicx}
\usepackage{wrapfig}

\usepackage{hyperref}
\usepackage[rgb]{xcolor}
\hypersetup{				% Гиперссылки
    colorlinks=true,       	% false: ссылки в рамках
	urlcolor=blue          % на URL
}

%  Русский язык

\usepackage[T2A]{fontenc}			% кодировка
\usepackage[utf8]{inputenc}			% кодировка исходного текста
\usepackage[english,russian]{babel}	% локализация и переносы


% Математика
\usepackage{amsmath,amsfonts,amssymb,amsthm,mathtools} 


\usepackage{wasysym}

%Заговолок
\author{Волков Сергей, Б06-806}
\title{Prefix-функция, Z-функция и алгоритм Кнута-Морриса-Пратта}
\date{\today}


\begin{document} % начало документа

\maketitle

\tableofcontents

\newpage

\section{Prefix-функция}

\subsection{Описание}

Prefix-функция – функция двух аргументов – строки ($s$) и номера ($i$) – выдающая на выходе число – длину максимально больших совпадающих собственных (то есть их длина меньше длины самой строки) префикса и суффикса строки $s[ : i+1]$.\\

Математически это можно выразить так:

\begin{equation}
\pi[i] = \max_{k = 0 \dots i-1} { \lbrace k : s[0 : k+1] = s[i-k+1 : i+1] \rbrace }
\end{equation}

\subsection{Пример расчета prefix-функции для строки}

Строка – <<abacaba>>.\\

\begin{table}[h!]

\begin{tabular}{|p{2.5cm}||p{1.3cm}|p{1.3cm}|p{1.3cm}|p{1.3cm}|p{1.3cm}|p{1.3cm}|p{1.3cm}|p{2.5cm}|}

\hline 
Элементы строки & a & b & a & c & a & b & a \\ 
\hline 
Значения prefix-функции & 0 & 0 & 1 & 0 & 1 & 2 & 3 \\ 
\hline 
Комментарии & Нет совпадений & Нет совпадений & a=a & Нет совпадений & a=a & ab=ab & aba= aba \\ 
\hline 
\end{tabular} 

\end{table}

\subsection{Тривиальный алгоритм}

Тривиальный алгоритм – расчет значения prefix-функции для каждого элемента строки в отдельности.\\

Асимптотика алгоритма – $O(n^3)$, т.к. всего $O(n^2)$ итераций цикла, на каждой из который происходит сравнение строк за $O(n)$, что дает в итоге $O(n^3)$.\\

\newpage

Реализация на Python 3.7:

\begin{lstlisting}[language=Python]

def prefix_trivial(string):

	prefix_values = [0] * len(string) ## list of  values
	for i in range(len(string):
		for k in range(i):
			if string[:k] == s[i-k, i]:
				prefix_values[i] = k			
	return prefix_values

\end{lstlisting}

\subsection{Оптимальный алгоритм}

В основе оптимального алгоритма расчета prefix-функции лежит такая идея:\\

Если в данный момент времени рассчитано значение  $\pi[i]$ и требуется рассчитать $\pi[i+1]$, то можно использовать значение на i-ом элементе:\\

Посмотрим на элемент $s[\pi[i]]$, и в том случае, когда он равен $s[i+1]$, значение $\pi[i+1]=\pi[i]+1$.\\

Если эти элементы не равны, то следует посмотреть на префикс максимальной длины на $s[ : \pi[i]]$ и сравнить следующий элемент после этого префикса с $s[i+1]$, если они окажутся равны, то значение $\pi[i+1]=length+1$, где $length$ – длина максимально большого префикса.\\

Если эти элементы опять различны, то повторяем алгоритм, пока не найдем равные элементы или не дойдем до $s[0]$.\\

Асимптотика алгоритма – $O(n)$. Это связано с тем, что при вычислении $\pi[i+1]$ мы делаем линейное количество сравнений [не более, чем $n-1$ в худшем случае ($n$ – длина строки)], например, в случае <<aaaab>> для символа b.\\

\newpage

Реализация на Python 3.7:

\begin{lstlisting}[language=Python]

def prefix(string):

    prefix_values = [0] * len(string) ## list of values
    k = 0
    for i in range( 1, len(string) ):
        while k > 0 and string[k] != string[i]: 
            k = prefix_values[k - 1]
        if string[k] == string[i]: 
            k += 1            
        prefix_values[i] = k
    return prefix_values

\end{lstlisting}

\section{Z-функция}

\subsection{Описание}

Z-функция – функция двух аргументов – строки ($s$) и номера ($i$) – выдающая на выходе число – длину максимально больших совпадающих собственных (то есть их длина меньше длины самой строки) префиксов строк $s$ и $s[i : ]$.\\

Математическое описание:

\begin{equation}
z[i] = \max_{k = 0 \dots len(s)-i-1} { \lbrace k : s[0 : k+1] = s[i : i+k+1] \rbrace }
\end{equation}

\subsection{Пример расчета Z-функции для строки}

Строка – <<abacaba>>. $z[0]=0$ по договоренности (доопределение).

\begin{table}[h!]

\begin{tabular}{|p{2.5cm}||p{1.3cm}|p{1.3cm}|p{1.3cm}|p{1.3cm}|p{1.3cm}|p{1.3cm}|p{1.3cm}|p{2.5cm}|}

\hline 
Элементы строки & a & b & a & c & a & b & a \\ 
\hline 
Значения Z-функции & 0 & 0 & 1 & 0 & 3 & 0 & 1 \\ 
\hline 
Комментарии & Нет совпадений & Нет совпадений & a=a & Нет совпадений & aba= aba & Нет совпадений & a=a \\ 
\hline 
\end{tabular} 

\end{table}

\subsection{Тривиальный алгоритм}

Тривиальный алгоритм – расчет значения Z-функции для каждого элемента строки в отдельности.\\

Асимптотика алгоритма – $O(n^3)$, т.к. всего $O(n^2)$ итераций цикла, на каждой из который происходит сравнение строк за $O(n)$, что дает в итоге $O(n^3)$.\\

Реализация на Python 3.7:

\begin{lstlisting}[language=Python]

def z_func_trivial(string):

	z_values = [0] * len(string)
	for i in range(1, len(string)):
		while (i+z_values[i] < len(string) and
		       string[z_values[i]] == string[i+z_values[i]]):
			z_values[i] += 1
	return z_values

\end{lstlisting}

\subsection{Оптимальный алгоритм}

В основе оптимального алгоритма расчета Z-функции лежит такая идея:\\

Если мы для некоторого $i$ нашли префикс длины такой, что для нескольких дальнейших итераций будет выполнено, что номер итерации меньше, чем номер правой границы получившегося префикса, то мы можем заполнить часть дальнейших значений Z-функции значениями из соответствующего префикса строки $s$.\\

Это можно сделать для всех $k=i-left:z[k]<right-i+1$, где $left, right$ – левая и правая границы т.н. Z-блока для i-го элемента. В случае, если это условие не выполнено, префикс k-го элемента может выходить за границы Z-блока, и нужно провести сравнение следующих элементов.\\

Асимптотика алгоритма – $O(n)$, доказательство аналогично доказательству линейности времени работы prefix-функции.

\newpage

Реализация на Python 3.7:

\begin{lstlisting}[language=Python]

def z_func(string):
    
    z_values = [0] * len(string)
    left, right = 0, 0
    for i in range(1, len(string)):
        if i > right:
            left = right = i
            while (right < len(string) and 
            	   string[right] == string[right - left]):
                right += 1
            z_values[i] = right - left
            right -= 1
        else: ## checking values from [left, right] interval
            j = i - left
            if z_values[j] < right - i + 1: ## if it's not 
                z_values[i] = z_values[j] ## last element
            else: ## if it's last element, compare next ones
                left = i
                while (right < len(string) and 
                	   string[right] == string[right - left]):
                    right += 1
                z_values[i] = right - left
                right -= 1      
    return z_values

\end{lstlisting}

\section{Алгоритм Кнута-Морриса-Пратта (КМП)}

\subsection{Описание алгоритма}

Алгоритм КМП нужен для поиска подстроки в строке, с его помощью можно не только определить, присутствует ли искомая подстрока в заданной строке, но и также сказать, на какой позиции в строке она находится.\\

Реализация алгоритма КМП состоит в применении prefix- или Z-функции на строке вида $target+@+string$, где $target$ – искомая подстрока, $string$ – заданная строка, а $@$ – некоторый символ, не встречающийся ни в $target$, ни в $string$. Этот символ нужен для того, чтобы значение prefix- или Z-функции преждевременно не достигло значения равного либо даже большего, чем $length(target)$.\\

Если на некотором элементе такой строки в области $string$ значение prefix- или Z-функции окажется равным длине $target$, то тогда $target$ входит в $string$. Для prefix-функции место, где $\pi[i]=length(target)$, определяет конец вхождения $target$ в $string$, а для Z-функции – начало вхождения. Соответственно для prefix-функции элемент под номером $i-length(target)$ будет началом вхождения, а для Z-функции элемент под номером $i+length(target)$ будет концом вхождения. Чтобы найти место вхождения в исходную строку, нужно вычесть (прибавить) соответствующие значения (см. реализацию).\\

Реализация на Python 3.7:

\begin{lstlisting}[language=Python]

def kmp_prefix(string, target):

	new_string = target + '@' + string
	enterings = []
	prefix_list = prefix(new_string)
	for i in range(len(prefix_list)):
		if prefix_list[i] == len(target):
			## appending entering of target
			## counting string from 0
			enterings.append(
			(i - 2*len(target), i - len(target) - 1) )
	return enterings

def kmp_z(string, target):

	new_string = target + '@' + string
	enterings = []
	z_list = z_func(new_string)
	for i in range(len(z_list)):
		if z_list[i] == len(target):
			## appending entering of target
			## counting string from 0
			enterings.append(
			(i - len(target) - 1, i - 2) )
	return enterings

\end{lstlisting}

\end{document}